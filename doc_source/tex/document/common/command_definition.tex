\newcommand{\setup}[2]{
%----------------------------------------------------------------------------------------
%	PACKAGES AND OTHER DOCUMENT CONFIGURATIONS
%----------------------------------------------------------------------------------------

    \documentclass[a4paper, 11pt, oneside]{book} % A4 paper size, default 11pt font size and oneside for equal margins

    \usepackage[margin=1in,nomarginpar]{geometry}
    \usepackage[utf8]{inputenc} % Required for inputting international characters
    \usepackage[T1]{fontenc} % Output font encoding for international characters
    \usepackage{fouriernc} % Use the New Century Schoolbook font
    \usepackage{titletoc}% http://ctan.org/pkg/titletoc
    \usepackage{hyperref}
    \usepackage{graphicx}
    \usepackage{lipsum}
    \usepackage{caption}
    \usepackage{float}
    \usepackage{amssymb}
    \usepackage{amsmath}

    \usepackage{xlop}
    \usepackage{tikz}
    \usepackage{xfp}
    \usepackage{soul}
    \usepackage{xifthen}
    \sodef\carrystyle{}{0.45em}{0em}{-0.25em}

    \newcommand{\opandbin}[3]{
        \oplogicalbin{##1}{##2}{##3}{$\land$}
    }

    \newcommand{\oporbin}[3]{
        \oplogicalbin{##1}{##2}{##3}{$\lor$}
    }

    \newcommand{\oplogicalbin}[4] {
        \begin{tabular}[t]{rr}
            &\pgfmathbin{##1}\pgfmathresult\\
            ##4&\pgfmathbin{##2}\pgfmathresult\\\hline
            &\pgfmathbin{##3}\pgfmathresult
        \end{tabular}
    }

    \newcommand{\opaddbin}[3]{
        \newcommand{\resultInBaseTen}{\fpeval{##1+##2}}

        \newcount\currentValue
        \newcount\basenumberlength
        \currentValue=\resultInBaseTen
        \basenumberlength=0

        \loop
        \advance \basenumberlength + 1
        \currentValue=\numexpr \currentValue / 2\relax
        \ifnum \currentValue > 1
        \repeat

        \pgfmathsetbasenumberlength{\basenumberlength}
        \begin{tabular}[t]{rr}
            \ifthenelse{\isempty{##3}}%
            {}% if #3 is empty
            {&\pgfcarrybin{##3}\\}% if #3 is not empty
            &\pgfmathbin{##1}\pgfmathresult\\
            +&\pgfmathbin{##2}\pgfmathresult\\\hline
            &\pgfmathbin{\resultInBaseTen}\pgfmathresult
        \end{tabular}
    }

    \newcommand{\pgfcarry}[1]{
        \begin{tiny}
            \expandafter\carrystyle\expandafter{##1}
        \end{tiny}
    }

    \newcommand{\pgfcarrybin}[1] {
        \pgfmathbin{##1}
        \pgfcarry{\pgfmathresult}
    }

    \usepackage{a4wide}
    \usepackage[#1]{babel}
    \usepackage[#2]{csquotes}
    \newcommand*{\itenquote}[1]{\enquote{{\itshape##1}}}
    \newcommand{\para}{\\[12pt]}
    \usepackage{subcaption}
    \usepackage[local,nolabels,exerciseaslist,usesolutionserieslabels]{exsol}
    \renewcommand{\loadSolutions}{
        \immediate\closeout\solutionstream
        \input{\jobname.sol.tex}
        \immediate\openout\solutionstream=\jobname.sol.tex
    }

    \setlength{\exsolexercisetopbottomsep}{0pt plus 0pt minus 1pt}
    \setlength{\exsolexerciseleftmargin}{2em}
    \setlength{\exsolexerciserightmargin}{1em}
    \setlength{\exsolexerciseparindent}{0em}
    \setlength{\exsolexerciselabelsep}{1ex}
    \setlength{\exsolexerciselabelwidth}{30pt}
    \setlength{\exsolexerciseitemindent}{0pt}
    \setlength{\exsolexerciseparsep}{\parskip}

    \usepackage[postpunc={dot},% full stop after description
    nostyles,% don't load default style packages
% load glossaries-extra-stylemods.sty and glossary-tree.sty:
    stylemods={tree}
    ]{glossaries-extra}

    \usepackage[backend=biber,
    style=alphabetic,
    ]{biblatex}

    \usepackage[nottoc]{tocbibind}
    \usepackage{listings}
    \usepackage{xcolor}

    \colorlet{punct}{red!60!black}
    \definecolor{background}{HTML}{EEEEEE}
    \definecolor{delim}{RGB}{20,105,176}
    \colorlet{numb}{magenta!60!black}

    \lstdefinelanguage{json}{
        basicstyle=\normalfont\ttfamily,
        numbers=left,
        numberstyle=\scriptsize,
        stepnumber=1,
        numbersep=8pt,
        showstringspaces=false,
        breaklines=true,
        frame=lines,
        backgroundcolor=\color{background},
        literate=
        *{0}{{{\color{numb}0}}}{1}
        {1}{{{\color{numb}1}}}{1}
        {2}{{{\color{numb}2}}}{1}
        {3}{{{\color{numb}3}}}{1}
        {4}{{{\color{numb}4}}}{1}
        {5}{{{\color{numb}5}}}{1}
        {6}{{{\color{numb}6}}}{1}
        {7}{{{\color{numb}7}}}{1}
        {8}{{{\color{numb}8}}}{1}
        {9}{{{\color{numb}9}}}{1}
        {:}{{{\color{punct}{:}}}}{1}
        {,}{{{\color{punct}{,}}}}{1}
        {\{}{{{\color{delim}{\{}}}}{1}
        {\}}{{{\color{delim}{\}}}}}{1}
        {[}{{{\color{delim}{[}}}}{1}
        {]}{{{\color{delim}{]}}}}{1},
}

% Table of contents with links
\hypersetup{
colorlinks,
citecolor=black,
filecolor=black,
linkcolor=black,
urlcolor=black
}
}