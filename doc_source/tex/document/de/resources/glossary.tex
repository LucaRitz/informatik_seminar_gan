\newglossaryentry{GAN} {
    name = {GAN},
    description = {Generative Adversarial Networks - Zwei neuronale Netze, die ein Nullsummenspiel durchführen.
    Eines erstellt Kandidaten, das andere bewertet sie.\cite{wiki:gan}}
}
\newglossaryentry{Multilayer perceptron} {
    name = {Multilayer-Perceptron},
    plural = {Multilayer-Perceptronen},
    description = {Ein Multilayer-Perceptron ist eine Klasse von künstlichen neuronalen Netzen. Es besteht aus mindestens
    drei Layern: dem Input-Layer, einem Hidden-Layer und einem Output-Layer. Vielfach wird es auch als \glqq vanilla\grqq{}
        neuronales Netz bezeichnet.\cite{wiki:multilayerPerceptron}}
}
\newglossaryentry{KNN} {
    name = {KNN},
    description = {Siehe \Gls{Multilayer perceptron}}
}
\newglossaryentry{Backpropagation} {
    name = {Backpropagation},
    description = {Die Backpropagation beschreibt ein Verfahren, um neuronale Netze zu trainieren. Es basiert auf dem
    mittleren quadratischen Fehler und gehört zu der Gruppe der überwachten Lernverfahren.\cite{wiki:backpropagation}}
}
\newglossaryentry{CNN} {
    name = {Convolutional Neuronal Network},
    description = {Dabei handelt es sich um eine spezielle Architektur eines \Gls{KNN}. Es funktioniert nach dem Prinzip
        der Konvolution. Das heisst, die Gewichte auf den Kanten sind nicht flach, diese werden über Konvolutions-Filter
        definiert und viele Bildabschnitte teilen sich so die Gewichte.\cite{wiki:cnn}}
}
